\documentclass[a4paper, 11pt, uplatex]{jsarticle}
% \usepackage{amsmath, amssymb}
\usepackage{amsmath, newtxmath}
\usepackage[dvipdfmx]{graphicx, color}
\usepackage[abs]{overpic}
\usepackage[dvipdfmx]{pict2e}
\definecolor{spot}{cmyk}{1,0,0,0}

\begin{document}

\title{図表の配置}
\author{ほげほげおじさん}
\maketitle


--- ソースファイル例 --- \\
ここで図\ref{figA}を見てください.
図で赤く示されているのが...
\begin{figure}
	\centering
	\includegraphics[scale=0.5]{apple.png}
	\caption{あっぷる}
	\label{figA}
\end{figure}



\begin{figure}
	\centering
	\begin{overpic}[scale=0.5, pagebox=cropbox, clip, grid]{apple.png}
		\color{spot} \linethickness{3pt}
		\put(100,80){\vector(-1, 0){25}}
		\put(155,75){\Huge \sffamily Byte!}
	\end{overpic}
	\caption{あっぷる}
	\label{figB}
\end{figure}



\end{document}