\documentclass[a4paper, 11pt, uplatex]{jsarticle}
% \usepackage{titlesec}
% \titleformat*{\section}{}

\begin{document}

\title{情報科学レポート}
\author{岡部蒼太}
\date{2022/10/03}
\maketitle


\section{はじめに}
この文書は,ごく基本的なレポートや論文の例を示すものです.
実際にこのソースを入力してタイプセット(コンパイル)し,
タイトル,著者名,本文,見出し,箇条書きがどのように表示されるかを確認してみましょう.

\section{見出し}
この文書の先頭にはタイトル,著者名,日付が出力されています.
特定の日付を指定することもできます.

そして,セクションの見出しが出力されています。
セクションの番号は自動的に付きます。

\section{箇条書き}
以下は箇条書きの例です。これは番号を振らない箇条書きです。
\begin{itemize}
	\item ちゃお
	\item りぼん
	\item なかよ
\end{itemize}

これは番号をふる箇条書きです.
\begin{enumerate}
	\item 富士
	\item 鷹
	\item なすび
\end{enumerate}

\section{おわりに}
これは一段組の例ですが,二段組に変更することもできます.
解説文を読んで,このソースをいろいろと変更してみましょう.
\end{document}